\label{releasenotes}

The technical changelog is automatically generated from the Bareos bug tracking system, see \url{http://bugs.bareos.org/changelog_page.php}.

Please note, that some of the subreleases are only internal development releases.

Open issues for a specific version are shown at
\url{http://bugs.bareos.org/roadmap_page.php}.

The overview about new feature of a release are shown at
\url{https://github.com/bareos/bareos} and in the \nameref{index} of this document.

This chapter concentrates on things to do when updating an existing Bareos installation.

\warning{While all the source code is published on \elink{GitHub}{https://github.com/bareos/bareos}, the releases of packages on \url{http://download.bareos.org} is limited to the initial versions of a major release. Later maintenance releases are only published on \url{https://download.bareos.com}.}

\releasenoteSection{Bareos-17.2}

    %\item added \package{bareos-storage-ceph} also for \os{SLES}{12sp2} and \os{SLES}{12sp3}.


\releasenote{17.2.6}{

\begin{tabular}{p{0.2\textwidth} p{0.8\textwidth}}
Code Release      & 2018-06-21\\
Database Version  & 2171 (unchanged)\\
Release Ticket    & \ticket{916}\\
Url               & \releaseUrlDownloadBareosCom{17.2} \\
\end{tabular}

This release contains several bugfixes and enhancements. Excerpt:
\begin{itemize}
    \item added platforms: \os{Fedora}{27}, \os{Fedora}{28}, \os{openSUSE}{15.0}, \os{Ubuntu}{18.04} and \os{Univention}{4.3}.
    \item \os{Univention}{4.3}: fixes integration.
    \item \ticket{872} adapted to new Ceph API.
    \item \ticket{943} use \package{tirpc} if Sun-RPC is not provided.
    \item \ticket{964} fixes the predefined queries.
    \item \ticket{969} fixes a problem of restoring more files then selected in \bareosWebui/BVFS.
    \item \bareosDir: fixes for a crash after reload in the statistics thread (\ticket{695}, \ticket{903}).
    \item \command{bareos-dbcheck}: cleanup and speedup for some some of the checks.
    \item adapted for \postgresql 10.
    \item gfapi: stale file handles are treated as warnings
\end{itemize}

}

\releasenote{17.2.5}{

\begin{tabular}{p{0.2\textwidth} p{0.8\textwidth}}
Code Release      & 2018-02-16\\
Database Version  & 2171 (unchanged)\\
Release Ticket    & \ticket{910}\\
Url               & \releaseUrlDownloadBareosCom{17.2} \\
\end{tabular}

This release contains several bugfixes and enhancements. Excerpt:
\begin{itemize}
    \item \bareosFd is ready for \os{AIX}{7.1.0.0}.
    \item \nameref{VMwarePlugin} is also provided for \os{Debian}{9}.
    \item NDMP fixes
    \item Virtual Backup fixes
    \item \package{bareos-storage-droplet}: improvements
    \item \command{bareos-dbcheck} improvements and fixes: with older versions it could happen, that it destroys structures required by \bcommand{.bvfs_*}{}.
    \item \ticket{850} fixes a bug on \os{Univention}{}: fixes a problem of regenerating passwords when resyncing settings.
    \item \ticket{890} \bcommand{.bvfs_update}{} fix. Before there have been cases where it did not update the cache.
    \item \bcommand{.bvfs_lsdirs}{} make limit- and offset-option work correctly.
    \item \bcommand{.bvfs_lsdirs}{} show special directory (like \path|@bpipe@/|) on the same level as \path|/|.
    \item \ticket{895} added description to the output of \bcommand{show}{filesets}.
    \item \bareosWebui: Restore Browser fixes
    \begin{itemize}
        \item There was the possibility of an endless loop if the BVFS API delivers unexpected results. This has been fixed. See bugreports \ticket{887} and \ticket{893} for details.
        \item \ticket{905} fixes a problem with file names containing quotes.
    \end{itemize}
    \item \linkResourceDirective{Dir}{Client}{NDMP Block Size} changed type from \dt{Pint32} to \dt{Size32}. This should not affect any configuration, but is more consistent with other block size configuration directives.
\end{itemize}

}


\releasenote{17.2.4}{

\begin{tabular}{p{0.2\textwidth} p{0.8\textwidth}}
Code Release      & 2017-12-14\\
Database Version  & 2171\\
Release Ticket    & \ticket{861}\\
Url               & \releaseUrlDownloadBareosOrg{17.2} \\
                  & \releaseUrlDownloadBareosCom{17.2} \\
\end{tabular}

This release contains several enhancements. Excerpt:
\begin{itemize}
  \item Bareos Distribution (packages)
  \begin{itemize}
    \item \package{python-bareos} is included in the core distribution.
    \item \package{bareos-storage-droplet} is a storage backend for the droplet library.
        Most notably it allows backup and restores to a S3 environment.
        \betaSince{sd}{bareos-storage-droplet}{17.2.4}
    \item \package{bat} has been removed, see section \nameref{bat}.
    \item platforms:
    \begin{itemize}
        \item Windows Clients are still supported since Windows Vista.
        \item MacOS: added to build chain.
        \item \bareosFd is ready for HP-UX 11.31 (ia64).
        \item Linux Distribution: Bareos tries to provide packages for all current platforms. For details, refer to \nameref{sec:packages}.
    \end{itemize}
    \item Linux RPM packages: allow read access to /etc/bareos/ for all users (however, relevant files are still only readable for the user \user{bareos}).
        This allows other programs associated with Bareos to also use this directory.
  \end{itemize}

  \item Denormalization of the \dbtable{File} database table
  \begin{itemize}
    \item The denormalization of the \dbtable{File} database table leads to enormous performance improvements in installation, which covering a  lot of file (millions and more).
    \item For the denormalization the database schema must be modified.
          \warning{Updating the database to schema version $\geq$ 2170 will increase the required disk space.
                Especially it will require around twice the amount of the current database disk space during the migration.}
    \item The \dbtable{Filename} database table does no longer exists. Therefore the \bcommand{.bvfs_*}{} commands do no longer output the \dbcolumn{FilenameId} column.
  \end{itemize}

  \item NDMP\_NATIVE support has been added. This include the NDMP features DAR and DDAR. For details see \nameref{sec:NdmpNative}.
  \item Updated the package \package{bareos-vmware-plugin} to utilize the Virtual Disk Development Kit (VDDK) 6.5.x. This includes support for \vSphere 6.5 and the next major release (except new features) and backward compatible with \vSphere 5.5 and 6.0. For details see \nameref{VMwarePlugin}.
  \item Soft Quota: automatic quota grace period reset if a job does not exceed the quota.
  \item \command{bareos-dbcheck}: disable all interactive questions in batch mode.
  \item \bcommand{list}{files}: also show deleted files (accurate mode).
  \item \bcommand{list}{jobstatastics}: added.
  \item \bcommand{purge}{}: added confirmation.
  \item \bcommand{list}{volumes}: fix limit and offset handling.
  \item \ticket{629} Windows: restore directory attributes.
  \item \ticket{639} tape: fix block size handling, AWS VTL iSCSI devices
  \item \ticket{705} support for MySQL 5.7
  \item \ticket{719} allow long JSON messages (has been increased from 100KB to 2GB).
  \item \ticket{793} Virtual Backups: skip jobs with no files.
\end{itemize}

}


\releasenoteSection{Bareos-16.2}

\releasenote{16.2.8}{

\begin{tabular}{p{0.2\textwidth} p{0.8\textwidth}}
Code Release      & 2018-07-06\\
Database Version  & 2004 (unchanged)\\
Release Ticket    & \ticket{863}\\
Url               & \releaseUrlDownloadBareosCom{16.2} \\
\end{tabular}

This release contains several bugfixes and enhancements. Excerpt:
\begin{itemize}
   \item gfapi-fd Plugin
   \begin{itemize}
      \item Allow to use non-accurate backups with glusterfind
      \item Fix backups with empty glusterfind filelist.
      \item Explicitly close glfs fd on IO-open
      \item Don't reinitialize the connection to gluster
      \item Fix parsing of missing basedir argument
      \item Handle non-fatal Gluster problems properly
   \end{itemize}
   \item Reset JobStatus to previous JobStatus in status SD and FD loops to fix status all output
   \item Backport ceph: ported cephfs-fd and \command{cephfs_device} to new api
   \item \ticket{967} Windows: Symbolic links are now replaceable during restore
\end{itemize}
}

\releasenote{16.2.7}{

\begin{tabular}{p{0.2\textwidth} p{0.8\textwidth}}
Code Release      & 2017-10-09\\
Database Version  & 2004 (unchanged)\\
Release Ticket    & \ticket{836}\\
Url               & \releaseUrlDownloadBareosCom{16.2} \\
\end{tabular}

This release contains several bugfixes and enhancements. Excerpt:
\begin{itemize}
    \item Fixes a Director crash, when enabling debugging output
    \item \bcommand{.bvfs_lsdirs}{}: improve performance, especially when having a large number of directories
    \begin{itemize}
      \item To optimize the performance of the SQL query used by \bcommand{.bvfs_lsdirs}{}, it is important to
      have the following indexes:
      \item PostgreSQL
        \begin{itemize}
          \item \sqlcommand{CREATE INDEX file_jpfnidpart_idx ON File(PathId,JobId,FilenameId) WHERE FileIndex = 0;}
          \item If the index \sqlcommand{file_jfnidpart_idx} mentioned in 16.2.6 release notes exist, drop it:\\
            \sqlcommand{DROP INDEX file_jfnidpart_idx;}
        \end{itemize}
      \item MySQL/MariaDB
        \begin{itemize}
          \item \sqlcommand{CREATE INDEX PathId_JobId_FileNameId_FileIndex ON File(PathId,JobId,FilenameId,FileIndex);}
          \item If the index \sqlcommand{PathId_JobId_FileIndex_FileNameId} mentioned in 16.2.6 release notes exist, drop it:\\
            \sqlcommand{DROP INDEX PathId_JobId_FileIndex_FileNameId ON File;}
        \end{itemize}
    \end{itemize}
    \item Utilize OpenSSL $\geq$ 1.1 if available
    \item Windows: fixes silent upgrade (\command{winbareos-*.exe /S})
    \item Windows: restore attributes also on directories (not only on files)
    \item Fixes problem with SHA1 signature when compiled without OpenSSL (not relevant for bareos.org/bareos.com packages)
    \item Packages for openSUSE Leap 42.3 and Fedora 26 have been added.
    \item Packages for AIX and current HP-UX 11.31
\end{itemize}

}


\releasenote{16.2.6}{

\begin{tabular}{p{0.2\textwidth} p{0.8\textwidth}}
Code Release      & 2017-06-22\\
Database Version  & 2004 (unchanged)\\
Release Ticket    & \ticket{794}\\
Url               & \releaseUrlDownloadBareosCom{16.2} \\
\end{tabular}

This release contains several bugfixes and enhancements. Excerpt:
\begin{itemize}
  \item Prevent from director crash when using incorrect paramaters of \bcommand{.bvfs_*}{} commands.
  \item Director now closes all configuration files when reloading failed.
  \item Storage daemon now closes the network connection when MaximumConcurrentJobs reached.
  \item New directive \configdirective{LanAddress} was added to the Client and Storage Resources of the director to facilitate a network topology where client and storage are situated inside of a LAN, but the Director is outside of that LAN. See \nameref{LanAddress} for details.
  \item A Problem in the storage abstraction layer was fixed where the director picked the wrong storage daemon when multiple storages/storage daemons were used.
  \item The device spool size calculation when using secure erase was fixed.
  \item \bcommand{.bvfs_lsdirs}{} no longer shows empty directories from accurate jobs.
    \begin{itemize}
        \item \warning{This decreases performance if your environment has a large numbers of directories. Creating an index improves the performance.}
        %In Bareos 16.2.6 the SQL Query used by \bcommand{.bvfs_lsdirs}{} was changed to not show
        %empty directories from accurate jobs. It turned out that that the changed
        %query causes performance issues when larger amounts of directories were backed up.\\
        \item \postgresql
        \begin{itemize}
            \item When using PostgreSQL, creating the following partial improves the performance sufficiently:\\
                \sqlcommand{CREATE INDEX file_jfnidpart_idx ON File(JobId, FilenameId) WHERE FileIndex = 0;}

            \item Run following command to create the partial index:\\
                \path!su - postgres -c 'echo "CREATE INDEX file_jfnidpart_idx ON File(JobId, FilenameId) WHERE FileIndex = 0; ANALYZE File;" | psql bareos'!

        \end{itemize}
        \item \mysql
        \begin{itemize}
            \item When using MySQL or MariaDB, creating the following index improves the performance:\\
                \sqlcommand{CREATE INDEX PathId_JobId_FileIndex_FileNameId ON File(PathId,JobId,FileIndex,FilenameId);}

            \item Run following command to create the index:\\
                \path!echo "CREATE INDEX PathId_JobId_FileIndex_FileNameId ON File(PathId,JobId,FileIndex,FilenameId);" | mysql -u root bareos!

            \item  However, with larger amounts of directories and/or involved jobs, even with this index
                the performance of \bcommand{.bvfs_lsdirs}{} may still be insufficient. We are working on optimizing
                the SQL query for MySQL/MariaDB to solve this problem.
        \end{itemize}
    \end{itemize}

  \item Packages for Univention UCS 4.2 have been added.
  \item Packages for Debian 9 (Stretch) have been added.
  \item WebUI: The post install script of the bareos-webui RPM package for RHEL/CentOS was fixed, it no longer tries to run a2enmod which does not exist on RHEL/CentOS.
  \item WebUI: The login form no longer allows redirects to arbitrary URLs
  \item WebUI: The used ZendFramework components were updated from version 2.4.10 to 2.4.11.
  \item WebUI: jQuery was updated from version 1.12.4 to version 3.2.0., some outdated browsers like Internet Explorer 6-8, Opera 12.1x or Safari 5.1+ will no longer be supported, see \elink{jQuery Browser Support}{http://jquery.com/browser-support/} for details.
\end{itemize}

}

\releasenote{16.2.5}{

\begin{tabular}{p{0.2\textwidth} p{0.8\textwidth}}
Code Release      & 2017-03-03\\
Database Version  & 2004 (unchanged)\\
Release Ticket    & \ticket{734}\\
Url               & \releaseUrlDownloadBareosCom{16.2} \\
\end{tabular}

This release contains several bugfixes and enhancements. Excerpt:
\begin{itemize}
  \item NDMP: critical bugfix when restoring large files.
  \item truncate command allows to free space on disk storages (replaces an purged volume by an empty volume).
  \item Some fixes were added regarding director crashes, Windows backups (VSS), soft-quota reset and API (bvfs) problems.
  \item WebUI: handle file names containing special characters, hostnames starting with numbers and long logfiles.
  \item WebUI: adds translations for Chinese, Italian and Spanish.
\end{itemize}

}

\releasenote{16.2.4}{

\begin{tabular}{p{0.2\textwidth} p{0.8\textwidth}}
Code Release      & 2016-10-28\\
Database Version  & 2004 (unchanged)\\
Release Ticket    & \ticket{698}\\
Url               & \releaseUrlDownloadBareosOrg{16.2} \\
                  & \releaseUrlDownloadBareosCom{16.2} \\
\end{tabular}

First stable release of the Bareos 16.2 branch.

\begin{itemize}
\item Configuration
     \begin{itemize}
     \item Bareos packages contain the default configuration in \nameref{sec:ConfigurationSubdirectories}. Please read \nameref{sec:UpdateToConfigurationSubdirectories} before updating (make a copy of your configuration directories for your \bareosDir and \bareosSd before updating). Note: as the old configuration files are still supported, in most cases no changes are required.
     \item The default configuration does no longer name the \resourcetype{Dir}{Director} and \resourcetype{Sd}{Storage} resources after the systems hostname (\path|$HOSTNAME-dir| resp. \path|$HOSTNAME-sd|) but use \resourcename{Dir}{Director}{bareos-dir} resp. \resourcename{Sd}{Storage}{bareos-sd} as defaults. The prior solution had the disadvantage, that \path|$HOSTNAME-dir|\hide{$} has also been set on \bareosFd not running on the \bareosDir, which almost ever did require changing this setting. Also the new approach aligns better with \nameref{sec:ConfigurationSubdirectories}.
     \item Due to limitation of the build system, the default resource \resourcename{Dir}{FileSet}{Linux All} have been renamed to \resourcename{Dir}{FileSet}{LinuxAll} (no space between Linux and All).
     \item The configuration of the \package{bareos-traymonitor} has also been split into resource files.
        Additional, these resource files are now packaged in other packages:
        \begin{itemize}
            \item \path|$CONFIGDIR/tray-monitor.d/monitor/bareos-mon.conf|: \package{bareos-traymonitor}
            \item \path|$CONFIGDIR/tray-monitor.d/client/FileDaemon-local.conf|: \package{bareos-filedaemon}
            \item \path|$CONFIGDIR/tray-monitor.d/storage/StorageDaemon-local.conf|: \package{bareos-storage}
            \item \path|$CONFIGDIR/tray-monitor.d/director/Director-local.conf|: \file{bareos-director}
        \end{itemize}
         This way, the \package{bareos-traymonitor} will be configured automatically for the installed components.
     \end{itemize}
\item Strict ACL handling
     \begin{itemize}
     \item Bareos Console \dt{Acl}s do no longer automatically matches substrings
        (to avoid that e.g. \linkResourceDirectiveValue{Dir}{Console}{Pool ACL}{Full} also matches \pool{VirtualFull}).
        To configure the ACL to work as before, \linkResourceDirectiveValue{Dir}{Console}{Pool ACL}{.*Full.*} must be set.
        Unfortunately the \bareosWebui 15.2 \resourcename{Dir}{Profile}{webui} did use \linkResourceDirectiveValue{Dir}{Console}{Command ACL}{.bvfs*}, which is also no longer works as intended. Moreover, to use all of \bareosWebui 16.2 features, some additional commands must be permitted, so best use the new \resourcename{Dir}{Profile}{webui-admin}.
     \end{itemize}
\item \bareosWebui
     \begin{itemize}
     \item Updating from Bareos 15.2: Adapt \resourcename{Dir}{Profile}{webui} (from bareos 15.2) to allow all commands of \resourcename{Dir}{Profile}{webui-admin} (\linkResourceDirective{Dir}{Console}{Command ACL}).
     Alternately modify all \resourcetype{Dir}{Console}s currently using \resourcename{Dir}{Profile}{webui} to use \resourcename{Dir}{Profile}{webui-admin} instead.
     \item While RHEL 6 and CentOS 6 are still platforms supported by Bareos, the package \package{bareos-webui} is not available for these platforms, as the required ZendFramework 2.4 do require PHP $>=$ 5.3.17 (5.3.23). However, it is possible to use \package{bareos-webui} 15.2 against \package{bareos-director} 16.2. Also here, the profile must be adapted.
     \end{itemize}
\end{itemize}
}

\releasenoteSection{Bareos-15.2}

\releasenote{15.2.4}{

\begin{tabular}{p{0.2\textwidth} p{0.8\textwidth}}
Code Release      & 2016-06-10\\
Database Version  & 2004 (unchanged)\\
Release Ticket    & \ticket{641} \\
Url               & \releaseUrlDownloadBareosCom{15.2} \\
\end{tabular}

For upgrading from 14.2, please see release notes for 15.2.1.

This release contains several bugfixes and enhancements. Excerpt:
\begin{itemize}
 \item Automatic mount of disks by SD
 \item NDMP performance enhancements
 \item Windows: sparse file restore
 \item Director memory leak caused by frequent bconsole calls
\end{itemize}
}


\releasenote{15.2.3}{

\begin{tabular}{p{0.2\textwidth} p{0.8\textwidth}}
Code Release      & 2016-03-11\\
Database Version  & 2004 (unchanged)\\
Release Ticket    & \ticket{625} \\
Url               & \releaseUrlDownloadBareosCom{15.2} \\
\end{tabular}

For upgrading from 14.2, please see releasenotes for 15.2.1.

This release contains several bugfixes and enhancements. Excerpt:
\begin{itemize}
 \item VMWare plugin can now restore to VMDK file 
 \item Ceph support for SLES12 included
 \item Multiple gfapi and ceph enhancements 
 \item NDMP enhancements and bugfixes
 \item Windows: multiple VSS Jobs can now run concurrently in one FD, installer fixes
 \item bpipe: fix stderr/stdout problems
 \item reload command enhancements (limitations eliminated)
 \item label barcodes now can run without interaction 
\end{itemize}
}



\releasenote{15.2.2}{

\begin{tabular}{p{0.2\textwidth} p{0.8\textwidth}}
Code Release      & 2015-11-19\\
Database Version  & 2004\\
                  & Database update required (if coming from bareos-14.2). See the \nameref{bareos-update} section.\\
Release Ticket    & \ticket{554} \\
Url               & \releaseUrlDownloadBareosOrg{15.2} \\
                  & \releaseUrlDownloadBareosCom{15.2} \\
\end{tabular}

First stable release of the Bareos 15.2 branch.

When coming from bareos-14.2.x, the following things have changed (same as in bareos-15.2.1):
\begin{itemize}
    \item The default setting for the Bacula Compatbile mode in  \linkResourceDirective{Fd}{Client}{Compatible} and \linkResourceDirective{Sd}{Storage}{Compatible} have been changed from \argument{yes} to \argument{no}.
    \item The configuration syntax for Storage Daemon Cloud Backends Ceph and GlusterFS have changed.
	Before bareos-15.2, options have been configured as part of the \linkResourceDirective{Sd}{Device}{Archive Device} directive, while now the Archive Device contains only information text and options are defined via the \linkResourceDirective{Sd}{Device}{Device Options} directive. See examples in \linkResourceDirective{Sd}{Device}{Device Options}.
\end{itemize}

}



\releasenoteUnstable{15.2.1}{

\begin{tabular}{p{0.2\textwidth} p{0.8\textwidth}}
Code Release      & 2015-09-16\\
Database Version  & 2004\\
                  & Database update required, see the \nameref{bareos-update} section.\\
Release Ticket    & \ticket{501} \\
Url               & \releaseUrlDownloadBareosOrg{15.2} \\
\end{tabular}

Beta release.

\begin{itemize}
    \item The default setting for the Bacula Compatbile mode in  \linkResourceDirective{Fd}{Client}{Compatible} and \linkResourceDirective{Sd}{Storage}{Compatible} have been changed from \argument{yes} to \argument{no}.
    \item The configuration syntax for Storage Daemon Cloud Backends Ceph and GlusterFS have changed.
	Before bareos-15.2, options have been configured as part of the \linkResourceDirective{Sd}{Device}{Archive Device} directive, while now the Archive Device contains only information text and options are defined via the \linkResourceDirective{Sd}{Device}{Device Options} directive. See examples in \linkResourceDirective{Sd}{Device}{Device Options}.
% # Old syntax:
% #    Archive Device = /etc/ceph/ceph.conf:poolname
% #
% # New syntax:
% #    Archive Device = <text>
% #    Device Options = "conffile=/etc/ceph/ceph.conf,poolname=poolname"
\end{itemize}

}


\releasenoteSection{Bareos-14.2}

It is known, that \command{drop_database} scripts will not longer work on PostgreSQL $<$ 8.4. However, as \command{drop_database} scripts are very seldom needed, package dependencies do not yet enforce PostgreSQL $>=$ 8.4.
We plan to ensure this in future version of Bareos.

\releasenote{14.2.7}{

\begin{tabular}{p{0.2\textwidth} p{0.8\textwidth}}
Code Release      & 2016-07-11\\
Database Version  & 2003 (unchanged)\\
Release Ticket    & \ticket{584} \\
Url               & \releaseUrlDownloadBareosCom{14.2} \\
\end{tabular}

This release contains several bugfixes. Excerpt:
\begin{itemize}
    \item bareos-dir
    \begin{itemize}
        \item Fixes pretty printing of Fileset options block \\
            \ticket{591}: config pretty-printer does not print filesets correctly
        \item run command: fixes changing the pool when changing the backup level in interactive mode \\
            \ticket{633}: Interactive run doesn't update pool on level change
        \item Ignore the Fileset option DriveType on non Windows systems \\
            \ticket{644}: Setting DriveType causes empty backups on Linux
        \item Suppress already queued jobs for disabled schedules \\
            \ticket{659}: Suppress already queued jobs for disabled schedules
    \end{itemize}
    \item NDMP
    \begin{itemize}
        \item Fixes cancel of NDMP jobs\\
            \ticket{604}: Cancel a NDMP Job causes the sd to crash
    \end{itemize}
    \item bpipe-fd plugin
    \begin{itemize}
        \item Only take stdout into account, ignore stderr (like earlier versions) \\
            \ticket{632}: fd-bpipe plugin merges stderr with stdout, which can result in corrupted backups
    \end{itemize}
    \item win32
    \begin{itemize}
        \item Fix symlink and junction support\\
            \ticket{575}: charset problem in symlinks/junctions windows restore \\
            \ticket{615}: symlinks/junctions wrong target path on restore (wide chars)
        \item Fixes quoting for bmail.exe in bareos-dir.conf \\
            \ticket{581}: Installer is setting up a wrong path to bmail.exe without quotes / bmail not called
        \item Fix crash on restore of sparse files \\
            \ticket{640}: File daemon crashed after restoring sparse file on windows
    \end{itemize}
    \item win32 mssql plugin
    \begin{itemize}
        \item Allow connecting to non default instance \\
            \ticket{383}: mssqldvi problem with connection to mssql not default instance
        \item Fix backup/restore of incremental backups \\
            \ticket{588}: Incremental MSSQL backup fails when database name contains spaces
    \end{itemize}
\end{itemize}
}

\releasenote{14.2.6}{

\begin{tabular}{p{0.2\textwidth} p{0.8\textwidth}}
Code Release      & 2015-12-03\\
Database Version  & 2003 (unchanged)\\
Release Ticket    & \ticket{474} \\
Url               & \releaseUrlDownloadBareosCom{14.2} \\
\end{tabular}

This release contains several bugfixes.
}

\releasenote{14.2.5}{

\begin{tabular}{p{0.2\textwidth} p{0.8\textwidth}}
Code Release      & 2015-06-01\\
Database Version  & 2003 (unchanged)\\
Release Ticket    & \ticket{447} \\
Url               & \releaseUrlDownloadBareosCom{14.2} \\
\end{tabular}

This release contains several bugfixes and added the platforms \os{Debian}{8} and \os{Fedora}{21}.
}


\releasenote{14.2.4}{

\begin{tabular}{p{0.2\textwidth} p{0.8\textwidth}}
Code Release      & 2015-03-23 \\
Database Version  & 2003 (unchanged)\\
Release Ticket    & \ticket{420} \\
Url               & \releaseUrlDownloadBareosCom{14.2} \\
\end{tabular}

This release contains several bugfixes, including one major bugfix (\ticket{437}), relevant for those of you using backup to disk with autolabeling enabled.

It can lead to loss of a 64k block of data when all of this conditions apply:
\begin{itemize}
 \item backups are written to disk (tape backups are not affected)
 \item autolabelling is enabled
 \item a backup spans over multiple volumes
 \item the additional volumes are newly created and labeled during the backup
\end{itemize}
If existing volumes are used for backups spanning over multiple volumes, the problem does not occur.

We recommend to update to the latest packages as soon as possible.

If an update is not possible immediately,
autolabeling should be disabled and volumes should be labelled manually
until the update can be installed. 

If you are affected by the 64k bug, we recommend that you schedule a full backup after fixing the problem in order to get a
proper full backup of all files.
}

\releasenote{14.2.3}{

\begin{tabular}{p{0.2\textwidth} p{0.8\textwidth}}
Code Release      & 2015-02-02 \\
Database Version  & 2003 (unchanged)\\
Release Ticket    & \ticket{393}\\
Url               & \releaseUrlDownloadBareosCom{14.2} \\
\end{tabular}

}

\releasenote{14.2.2}{

\begin{tabular}{p{0.2\textwidth} p{0.8\textwidth}}
Code Release      & 2014-12-12 \\
Database Version  & 2003 (unchanged)\\
                  & Database update required if updating from version $<$ 14.2.\\
                  & See the \nameref{bareos-update} section for details.\\
Url               & \releaseUrlDownloadBareosOrg{14.2} \\
                  & \releaseUrlDownloadBareosCom{14.2} \\
\end{tabular}

First stable release of the Bareos 14.2 branch.
}

\releasenoteUnstable{14.2.1}{

\begin{tabular}{p{0.2\textwidth} p{0.8\textwidth}}
Code Release & 2014-09-22 \\
Database Version  & 2003\\
                  & Database update required, see the \nameref{bareos-update} section.\\
Url               & \releaseUrlDownloadBareosOrg{14.2} \\
\end{tabular}

Beta release.
}

\releasenoteSection{Bareos-13.2}

\releasenote{13.2.5}{

\begin{tabular}{p{0.2\textwidth} p{0.8\textwidth}}
Code Release      & 2015-12-03 \\
Database Version  & 2002 (unchanged)\\
Url               & \releaseUrlDownloadBareosCom{13.2} \\
\end{tabular}

This release contains several bugfixes.
}

\releasenote{13.2.4}{

\begin{tabular}{p{0.2\textwidth} p{0.8\textwidth}}
Code Release      & 2014-11-05 \\
Database Version  & 2002 (unchanged)\\
Url               & \releaseUrlDownloadBareosCom{13.2} \\
\end{tabular}
}

\releasenote{13.2.3}{

\begin{tabular}{p{0.2\textwidth} p{0.8\textwidth}}
Code Release      & 2014-03-11 \\
Database Version  & 2002\\
                  & Database update required, see the \nameref{bareos-update} section.\\
Url               & \releaseUrlDownloadBareosCom{13.2} \\
\end{tabular}

It is known, that \command{drop_database} scripts will not longer work on PostgreSQL $<$ 8.4. However, as \command{drop_database} scripts are very seldom needed, package dependencies do not yet enforce PostgreSQL $>=$ 8.4. We plan to ensure this in future version of Bareos.
}

\releasenote{13.2.2}{

\begin{tabular}{p{0.2\textwidth} p{0.8\textwidth}}
Code Release      & 2013-11-19 \\
Database Version  & 2001 (unchanged)\\
Url               & \releaseUrlDownloadBareosOrg{13.2} \\
                  & \releaseUrlDownloadBareosCom{13.2} \\
\end{tabular}
}



\releasenoteSection{Bareos-12.4}


\releasenote{12.4.8}{

\begin{tabular}{p{0.2\textwidth} p{0.8\textwidth}}
Code Release      & 2015-11-18 \\
Database Version  & 2001 (unchanged)\\
Url               & \releaseUrlDownloadBareosCom{12.4} \\
\end{tabular}

This release contains several bugfixes.
}


\releasenote{12.4.6}{

\begin{tabular}{p{0.2\textwidth} p{0.8\textwidth}}
Code Release      & 2013-11-19 \\
Database Version  & 2001 (unchanged)\\
Url               & \releaseUrlDownloadBareosOrg{12.4} \\
                  & \releaseUrlDownloadBareosCom{12.4} \\
\end{tabular}
}



\releasenote{12.4.5}{

\begin{tabular}{p{0.2\textwidth} p{0.8\textwidth}}
Code Release      & 2013-09-10 \\
Database Version  & 2001 (unchanged)\\
Url               & \releaseUrlDownloadBareosCom{12.4} \\
\end{tabular}
}


\releasenote{12.4.4}{

\begin{tabular}{p{0.2\textwidth} p{0.8\textwidth}}
Code Release      & 2013-06-17 \\
Database Version  & 2001 (unchanged)\\
Url               & \releaseUrlDownloadBareosOrg{12.4} \\
                  & \releaseUrlDownloadBareosCom{12.4} \\
\end{tabular}
}


\releasenote{12.4.3}{

\begin{tabular}{p{0.2\textwidth} p{0.8\textwidth}}
Code Release      & 2013-04-15 \\
Database Version  & 2001 (unchanged)\\
Url               & \releaseUrlDownloadBareosOrg{12.4} \\
                  & \releaseUrlDownloadBareosCom{12.4} \\
\end{tabular}
}


\releasenote{12.4.2}{

\begin{tabular}{p{0.2\textwidth} p{0.8\textwidth}}
Code Release      & 2013-03-03 \\
Database Version  & 2001 (unchanged)\\
\end{tabular}
}

\releasenote{12.4.1}{

\begin{tabular}{p{0.2\textwidth} p{0.8\textwidth}}
Code Release      & 2013-02-06 \\
Database Version  & 2001 (initial)\\
\end{tabular}

This have been the initial release of Bareos.

Information about migrating from Bacula to Bareos are available at \elink{Howto upgrade from Bacula to Bareos}{http://www.bareos.org/en/HOWTO/articles/upgrade_bacula_bareos.html} and in section \nameref{compat-bacula}.
}
